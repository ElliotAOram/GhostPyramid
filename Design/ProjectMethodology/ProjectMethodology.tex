\documentclass{article}

\usepackage{cite}
\usepackage{textcomp}
\usepackage[T1]{fontenc}
\usepackage{fancyhdr}
\pagestyle{fancy}
\fancyhf{}
\lhead{Draft 0.2}
\rhead{Elliot Oram}
\rfoot{\thepage}


\title{Project Methodology}
\author{elo9@aber.ac.uk}

\begin{document}

\maketitle
\tableofcontents

\newpage

\section{Feature Driven Development}
Feature Driven Development (FDD) is a plan driven methodology for development in software engineering. FDD has well defined roles, such as Project Manager and Chief Architect, and develops a system by designing and building features. Features are normal single operations or functionalities that are desired by the Domain Expert (Customer). Several features are created during every iteration which is a period of time (normally 1 - 3 weeks) agreed upon at the begining of the project for when the next iteration of the software will be complete. FDD follows a five step plan with the final two steps being carried out repeatedly for each feature in the feature list and multiple times per iteration \cite{FDDandAgile}.

\begin{enumerate}
	\item \textbf{Develop an Overall Model}: At the start of a project, an overall model of the system is created with the Domain Expert present using Domain specific language. This model is designed to form the basis of how the system architecture is likely to be implemented. The model should cover all entities related to the system and will consist of various UML diagrams. FDD also uses colour UML, where different classes are coloured to represent real-world classifications of objects such as, a role in the system or an interval of time.
	\\
	\textbf{Roles}: Domain Experts \textbullet{} Chief Programmers \textbullet{} Chief Architect
	
	\item \textbf{Build a Feature List}: A feature is an isolated function that is required for the system to perform as the Domain Expert expects. A Feature should be written in Domain specific language and model a business activity. At this stage, all features are written into a list but have no weighting (prioritisation). Features are designed to take no longer than 2 weeks to develop. Features are written in the form: \\ 
\textbf{'<action> the <result> <by|for|of|to> a(n) <object>'}
\\An example of this is: \\
\textbf{'Display the contents of a shopping basket'}.
	\\
	\textbf{Roles}: Domain Experts \textbullet{} Chief Programmers \textbullet{} Chief Architect
	
	\item \textbf{Plan by Feature}: An overall development plan regarding teams to be assigned to each feature, feature prioritisation, scheduling of developer teams time and features to be completed in each iteration is produced. At this stage, the development teams are created, and the features to be developed are distributed amongst the chief programmers in charge of those teams. Furthermore, dependencies between features are established to ensure the correct sequence of development for the project.
	\\
	\textbf{Roles}: Project Manager \textbullet{} Chief Programmers \textbullet{} Chief Architect
	
\newpage %formats bullet 4/5 onto next page%

	\item \textbf{Design by Feature}: A single feature from the 'inbox' of features is selected by the chief programmer and then a design is produced for the implementation of that feature. The design for the feature will include specific UML diagrams as well as the creation of interfaces and function definitions. The overall design for the system will most likely be updated at this point if required.
	\\
	\textbf{Roles}: Chief Programmers \textbullet{} Development team
	
	
	\item \textbf{Build by Feature}: Once the feature has been successfully designed, the classes and methods associated with that feature can be implemented. This is followed by a code inspection and unit testing. Providing this is successful, the change is promoted to the build and becomes part of the clean code base ready to be shipped in the next iteration.
	\\
	\textbf{Roles}: Chief Programmers \textbullet{} Development team

\end{enumerate}

FDD is normally considered for larger projects as it provides a framework for distributed development. By dividing developers into smaller teams, FDD allows those teams to tackle features one at a time in parallel. Furthermore, the up front planning stage is generally indicative of projects that are more stable as, whilst it can be adapted throughout the process, the overall model is normally only added to, and the core architecture remains reasonably static.

As per the outline project specification, the steps to complete this project are well defined and I believe the project would be best suited to having some form of up front design. Furthermore, the continuous integration (CI) style of FDD offers a good way to produce a functional prototype at various stages of the project. CI will be a significant aid for the mid project review and potentially showcasing a prototype at this years science week.

\section{Single Person FDD}
To successfully use FDD for this project, I will have to make several adaptations to the normal processes of the methodology. The most notable is the developer team is only myself and therefore, I will be required to assume many different roles throughout the project.

\begin{itemize}
	\item \textbf{Stage 1 and 2}: Initially, a Domain Expert (Customer) is required to aide in the development of the overall model and feature creation. In this context, my supervisor can fulfil this role and I shall be both the Chief Programmer and Chief Architect. This project has the advantage of the Domain specific language being shared as both the Domain Expert and myself, as the developer, have the same computing background. The feature list produced at this stage will form the requirements list for the project.

\newpage

	\item \textbf{Stage 3}: Steps such as establishing developer teams and scheduling developer teams time throughout the project no longer exist. In their place, I will have to create a system to schedule by own time between the different aspects of the system. The main tasks to consider at this point will be the prioritisation of features and the scheduling of features per iteration. At this stage I will assume the roles of both the Project Manager drawing from the knowledge obtained of the system in Stages 1 and 2.

	\item \textbf{Stage 4 and 5}: The design and implementation of features will remain similar to the original specification. I will take the role of both the Developer Team and Chief Programmer and perform both the design and building of features. I will also update the overall model with the design for each feature when required. I will be using GitHub for version control and will create an issue that will correspond to each feature in the feature list. To allow for a code review/walk through, I plan to leave completed features as open pull requests for a day before reviewing the code. This should ensure good quality code is produced for each feature. This process will be followed for all pull requests unless they are urgent, or highly dependant on further development. I shall also create GitHub milestones to correspond to dates for each iteration. These milestones will contain a list of the features that are required for that iteration. Furthermore, this will help to show progress and if I am on track for the project deadlines.

\end{itemize}

\newpage

\section{Process details}
To ensure good quality code, I plan to incorporate several processes already used in FDD and other Agile methodologies.

\subsection{Test Driven Development(TDD)}
TDD will help ensure that code is written with the unit, and system, test functionality in mind. By writing the tests for the code first, I can be sure that the correct functionality is implemented and it meets the criteria of the feautre. Furthermore, testing will ensure that if a depending part of the code base changes, a test will fail and alert me to the problem. As I am using GitHub for this project, I hope to potentially pair this with CI tool for running tests such as Jenkins \cite{jenkins}. Jenkins will be able to run my tests automatically on every pull request I create. This will ensure that the code base is running correctly before being merged with the master branch.

\subsection{Coding standards}
By introducing coding standards, code readability and consistency will be improved. Furthermore, many accepted standards exist for languages already and Jenkins has tools that allow you to check both style and syntax for errors as well as coding standards. An example of this for python is Pylint \cite{pylint}. When run on python code, Pylint will produce a number of error, or warnings, showing where and how the code breaks the PEP 8 standard Pylint checks against \cite{pep8}. PEP 8 is produced by python themselves and therefore I believe it to be a sensible choice of coding standard to follow.

\subsection{Refactoring}
Refactoring will be a process that could potentially take place as frequently as after, or during, every feature. To ensure the quality and readability of the code, it is good practice to change or improve the way the code is written without changing the functionality. Refactoring, will often lead to a better more understandable code base thanks to the improved readability of the code. When paired with TDD, refactoring will be easy to do as the test can ensure that the functionality of the refactored code still adheres to the criteria of tests. Refactoring will take place as part of the TDD process using Red-Green-Refactor.


\nocite{*} % include everything from the bibliography, irrespective of whether it has been referenced.

\newpage

\bibliographystyle{IEEEtran}
\section{Bibliography}
\bibliography{IEEEabrv,ProjectMethodology} % References file


\end{document}
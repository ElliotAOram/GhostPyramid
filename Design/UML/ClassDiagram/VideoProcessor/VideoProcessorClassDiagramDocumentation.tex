\documentclass{article}

\usepackage[T1]{fontenc}
\usepackage{graphicx}
\usepackage{fancyhdr}
\pagestyle{fancy}
\fancyhf{}
\lhead{Draft 0.1}
\rhead{Elliot Oram}
\rfoot{\thepage}


\title{Video Processor Class Diagram}
\author{elo9@aber.ac.uk}

\begin{document}

\maketitle
\tableofcontents

\newpage

\section{Video Processor Class Diagram}
\includegraphics[width=200pt]{VideoProcessorClassDiagramImage}


\section{Description of Class Diagram}
The video processing will be performed by a single class, the \textbf{VideoProcessor}. The justification for doing this in a single class, is the remove the possibility that the Video object will have to passed between classes. In addition the functionality required for video feed manipulation doesn't have many stages.

\begin{itemize}

	\item \textbf{Video}: The video object is of type Video. This class type holds a video stream and can be imported from OpenCV. Whether this works for live video, needs to be investigated. To maximise the performance of the system and therefore the speed of processing, the video object may be treated as a global object. To allow for the preferred private video object, investigation into the representation of python objects will be required. This investigation will conclude if references or pointer to objects can be passed between functions, rather than the full object itself.

	\item \textbf{subtractBackground}: The subtractBackground function will be used to ensure that only the actor is shown in the video feed and the background is removed (set to black (RGB(0,0,0)). Based on research stated in the Outline Project Specification, This will use simple background subtraction or potentially moving average background subtraction. Spike solutions will be carried out during the creation of the prototype to assess if the image processing machine is capable of more complex background subtraction techniques.

	\item \textbf{transformVideo}: The transformVideo function will duplicate the video feed into 4 identical feeds and orientate them at 90 degrees from one another around a central point (This is the format that is expected for the pyramid to produce holograms).

	\item \textbf{outputVideo}: The outputVideo function will add the video feed to a window and this will be displayed in the Viewing Area to create the hologram.

\end{itemize}

\end{document}
\chapter{Third-Party Code and Libraries}

\section{Django}

A web frame work for creating Model View Controller style web applications using the Python programming language. Django works on a request based system where HTTP requests are routed to the controller which will then query the Model and View to create a rendered web page. This is then display in the client-side browser. 

Django version 1.8.3 was used as the web framework to implement the Charades Game within this project. For more information on Django, see the Django developer website %\cite{django_dev}

\section{Django-Jenkins}

django-jenkins is a plugin that captures the output of Django style unit and system tests and parses the output into a format that can be comprehended by Jenkins. This is a Django plugin that was developed by GitHub user kmmbvnr. This plugin is only a server side requirement so will be commented out from the final code base before final hand in of the technical work. Please see the README.txt file bundled with the source code to find out how to enable this functionality.

django-jenkins version 0.110.0 was used for this project. More information on the plugin can be found on the GitHub page %\cite{django-jenkins}

\section{OpenCV}

OpenCV is an open source computer vision library designed aid in building computer vision systems. The library is original written in C++, but can also be compiled for C, Java, Python and MATLAB. 
	
\section{SciPy}
 
SciPy text

\section{Selenium}

Selenium text

\section{JQuery}

JQuery text

\section{Jenkins}

Jenkins text


\section{NumPy}

NumPy text
\chapter{Code Examples}

\section{Embedded Python}

The code shown below is embedded Python code taken from the acting.html template of the Charades Game. The code below is designed to change the colour of each word in the current phrase depending on if it has been guessed, it is the current word or it has yet to be used.

\begin{itemize}
	\item The use of '{\% \%}' denote a Python block where the code inside is to be executed upon page load. 
	
	\item 'forloop.counter' and 'forloop.counter0' are internal counters used to index the current position in the for loop. 'forloop.counter0' is a counter starting from 0 and 'forloop.counter' is a counter starting from 1.

\end{itemize}

\begin{verbatim}
<h1 class="page_title">


<span class="current_word_span">

<span class="completed_word_span">

<span class="word_span">

{{word}}</span>

</h1>
\end{verbatim}

\section{phrase\_ready API}

The Charades Game API function that displays if the phrase is currently in a guessable state within the system. The function is from the 'charades/charades/views.py' and is discussed in Chapter 3: Iteration 7.

\begin{verbatim}
def phrase_ready_api(request):
    """
    Asserts if the phrase is ready. Only intended for api access
    """
    phrase_ready = False
    if GAME is not None:
        if GAME.actor is not None:
            phrase_ready = GAME.actor.phrase_ready()
    return render(request,
    		      'phrase_ready.html',
    		      {'phrase_ready' : phrase_ready})
\end{verbatim}


\newpage


\section{Polling function}

The JavaScript and JQuery function used to retrieve data from the guess\_correct API. The function is from the file '/charades/templates/guess.html' and discussed in Chapter 3: Iteration 7.

\begin{verbatim}
<script
src="https://ajax.googleapis.com/ajax/libs/jquery/3.2.0/jquery.min.js">
</script>
<script>
setInterval(function(){
    $.get( "/api/guess_correct", function( data ) {
        if (data.includes("Word") || data.includes("Phrase")) {
            window.location.replace('/waiting_for_actor/');
        }
    });
    poll();
}, 1000);
</script>
\end{verbatim}

\section{Unit test}

The below code shows the success, edge and failure states that are tested by unit tests. Here 0 is the edge case, 10 is the success case and -1 the failure.
This code is taken from the '/python/tests/test\_parsers.py' file.

\begin{verbatim}
class TestPositiveParser(unittest.TestCase):
    """
    @class TestPositiveParser :: Tests that ensure the input parser
    							 only accepts postive integers
    
    """
    ###---------------Success cases---------------###
    def test_zero(self):
        self.assertTrue(vpa.parse_positive_int(0))

    def test_positive(self):
        self.assertTrue(vpa.parse_positive_int(10))

    ###-------------Failure cases-----------------###

    def test_negative(self):
        self.assertRaises(ValueError,
                          vpa.parse_positive_int,
                          -1)

\end{verbatim}

\section{context file}

The context.py file is used to set the correct file path to import source code for the project. This allows project code to be executed in the tests. This code is from the file "/python/tests/context.py"

\begin{verbatim}
"""Allows tests to find the VideoProcessor module for testing
   This is required as the modules are in different directories"""
#pylint: disable=wrong-import-position,unused-import
#   Above pylint disables are required as this is a context file and
#   is only used by other files to set namespace and resovle modules.

import os
import sys
sys.path.insert(0, os.path.abspath(os.path.join(
			os.path.dirname(__file__), '..')))

import vpa

#http://docs.python-guide.org/en/latest/writing/structure/
\end{verbatim}
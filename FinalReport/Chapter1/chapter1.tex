\chapter{Background \& Objectives}

This section should discuss your preparation for the project, including background reading, your analysis of the problem and the process or method you have followed to help structure your work.  It is likely that you will reuse part of your outline project specification, but at this point in the project you should have more to talk about. 

\textbf{Note}: 

\begin{itemize}
   \item All of the sections and text in this example are for illustration purposes. The main Chapters are a good starting point, but the content and actual sections that you include are likely to be different.
   
   \item Look at the document on the Structure of the Final Report for additional guidance. 
   
\end {itemize}

\section{Background}
What was your background preparation for the project? What similar systems did you assess? What was your motivation and interest in this project?

The Pepper's Ghost technique was originally used for stage and theatre productions in the Victorian era to display holographic illusions to the audience. The technique, discovered by Dr. Henry Pepper, was first used in theatre in xxxx. The holograms being visible is reliant on lighting and the viewer being positioned correctly relative to a transparent surface. Figure 1 shows a basic example of how the technique produces holographic illusions. The object of interest is the focus of the lighting and, in this implementation, is out of the line of the sight of the audience. The light bounces from the object of interest and travels radially until it makes contact with the surrounding surfaces. The majority is absorbed by the surrounding black walls, but the remainder will reach the angled transparent plane. The plane is angled at a 45 degree offset to the object of interest. The plane manipulates the light particles, by refraction, to display them at a right angle to their angle of entry. The audience are positioned at 90 degrees from the object of interest and therefore, when the light is refracted, they see the object of interest appear in front of them. 

This technique has been used in different applications in following years such as [arcade machine] but it has had a more recent resurgence as an impactful display media in computer visualisation. Whilst the technique still uses Henry Pepper's original concept, it has been modified to display the hologram from multiple angles. This technique is now more commonly known as the Pepper's Ghost pyramid as it uses a transparent pyramid rather than a single plane. Figure 2 shows an example of a Pepper's Ghost pyramid. The pyramid is square based, made from a transparent material (such as perspex or clear acrylic) and is open at both the top and bottom. The technique for displaying holograms differs with the object of interest now being an image or video displayed on a screen. Furthermore, to create an illusion from all sides of the pyramid, four images are required (one for each side of the pyramid). This  means that the 2D projection of the image can be seen from any side of the pyramid. Figure 3 shows this in more detail. 

Whilst the pyramid implementation is an improvement on Pepper's original design it still suffers from the some of the same limitations of the original. The original design is reliant on the viewer looking directly at the transparent plane which makes it both vertically and horizontally intolerant for the viewer. The pyramid implementation means that the hologram is visible from four perspectives (in front, behind, left and right). Whilst this is an improvement, the hologram is not easily viewed from the edges of the pyramid and does not affect the vertical intolerance as viewing from above is not possible. Some concepts for a spherical design (instead of a pyramid) are proposed to resolve the horizontal intolerance completely. 

\section{Analysis}
Taking into account the problem and what you learned from the background work, what was your analysis of the problem? How did your analysis help to decompose the problem into the main tasks that you would undertake? Were there alternative approaches? Why did you choose one approach compared to the alternatives? 

There should be a clear statement of the objectives of the work, which you will evaluate at the end of the work. 

In most cases, the agreed objectives or requirements will be the result of a compromise between what would ideally have been produced and what was determined to be possible in the time available. A discussion of the process of arriving at the final list is usually appropriate.

As mentioned in the lectures, think about possible security issues for the project topic. Whilst these might not be relevant for all projects, do consider if there are relevant for your project. Where there are relevant security issues, discuss how they will this affect the work that you are doing. Carry forward this discussion into relevant areas for design, implementation and testing.

\section{Process}
You need to describe briefly the life cycle model or research method that you used. You do not need to write about all of the different process models that you are aware of. Focus on the process model that you have used. It is possible that you needed to adapt an existing process model to suit your project; clearly identify what you used and how you adapted it for your needs.


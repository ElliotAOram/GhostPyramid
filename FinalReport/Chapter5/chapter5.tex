\chapter{Evaluation}

\section{Methodology}
Choosing FDD for the project proved useful from a structural and time management perspective. Whilst multiple adaptations needed to be made to ensure the methodology was suitable for a single developer, the adapted methodology still captures the core concepts of the original specification. Developing features one at a time and having a set work flow for developing those features, helped to schedule tasks, and avoid being overwhelmed by the scale of the project. In addition, as FDD provided an upfront design phase it was far easier to obtain insight into the size of the project, as well as identifying areas that would take longer to implement.

Whilst the methodology was a success overall, it could have been further improved with additional adaptations. Originally, it was planned that pull requests would be left open on GitHub for a day, before they were reviewed to check for code quality. This in principal was a good idea but caused problems due to the dependencies between features. The feature list details that almost all features are dependent on the previous feature being complete. As such, waiting a day before performing a code review, would cause a delay in development. This problem could have been mitigated by developing the Charades Game system and the Hologram system in tandem. By doing this, it would have been possible to leave time to review one systems code while writing the other (hence not losing any development time).

\section{Requirements identification}

Since the initial creation of the objectives stated in chapter 1 section 1.4.1, some additional considerations have been added. Firstly, there is no mention of security explicitly in the objectives. The justification for this is that neither system produced in this project stores any personal data. Therefore, from a data security aspect, there is no requirement to protect user data. However, there are other security aspects to consider such as SQL injection or Cross-Site Request Forgery (CSRF). These forms of security threat have been addressed in the project although not stated in the objectives and in hindsight, it would have been better to have security explicitly stated as a requirement.

An additional objective that was not stated and should have been added was to ensure that the Charades Game UI was mobile compatible. The system itself was developed on a 17.3" monitor but was infrequently checked for how it would appear on a smaller (mobile) device. This would be an area of improvement that could be made going forward and would greatly benefit the users. By using an adaptive design based on screen size, it would be possible to have a display that would be correct for any size device.

Aside from the above statements, the requirements for the project were well written and helped to focus the project and ensure it did not go out of scope. All the project objectives that were set have been addressed in some way in the project. This system has been well written and has a very high test coverage, which suggests that the system is very robust. The software includes error checking at runtime (in the form of input parsers), as well as logical checks to ensure that any user input is valid and expected. The robustness of this system was confirmed at the Aberystwyth University Science Week 2017, where the system functioned without error for the duration of the event.


\section{Design decisions}

The Charades Game was originally designed as an android application as discussed in the implementation section. This design decision was made due to language and technology familiarity for the developer. After further analysis, during the implementation of the project, the decision to switch to a web based implementation was made. This initially resulted in the loss of two weeks' work as the Android application so far had to be deleted. This decision was overall a positive change, but did potentially lower the quality of the software due to the time that was lost.

Neither the Android application or web application had a comprehensive design for a method of inter user communication. As this technology was relatively unknown to the developer, it was difficult to decide the best method for implementation. Subsequently the communications design was vague. With more investigation into possible technologies for communication in the earlier stages of design, it would have been easier to make a more informed choice of the type of technology to use.

The design diagrams used to inform the overall model (section 2.1) proved very useful in helping to define the system. All the diagrams provided good insight into the requirements and structure of the system and were used throughout to help meet those requirements. Starting with a use case diagram helped to view the system at a very basic level of interaction. It helped to formalise all the operations that would be required of the system and ensured that no functional requirements were missed.

Creating UI designs helped to ensure that functions identified from the Use Case diagram were met. However, the diagrams could have been improved to help with system interaction. Being planned for use at outreach events, the system had to be designed with a young target audience in mind. In this respect, the UI design falls slightly short as the colour scheme could be more vibrant and enticing, and the interactions could have been designed in a way where instructions (displayed on the user and actor landing pages) are not required. Having not been able to test this system with the target audience, it is hard to establish if this is an effective UI.

\section{Use of Tools}
When the hologram system was being developed, it used the notepad++ text editor as the main editor for Python coding. However, when development of the Charades Game began, it required a better IDE that could help when developing code. The PyCharm IDE proved very helpful while developing the Charades System and on reflection would have been an excellent tool to use for development of the hologram system as well. Whilst it is unlikely that this would have further improved the code quality beyond its current state, it would have reached the state faster.

Throughout the project, Git and GitHub were utilised well making it easy to track progress, and organise features. The extensive use of Git along with sensible commit messages and correct issue number referencing made it easy to look back and see what development had been completed already. Furthermore, the use of GitHub templates for issue and pull request creation meant that a check list was already generated for each issue and pull request which contained the tasks to be completed. Although GitHub was an asset to the project, the way in which it was used, at times, was excessive and could get in the way of development. The GitHub work flow catered towards a team slightly more than towards an individual. This meant that unnecessary tasks like assigning a developer to an issue or pull request had to be performed despite there only being one developer. Despite this, Git and GitHub were invaluable tools for version control, organisation, and backup throughout the project. 

Jenkins was used as an environment to run the automated tests for the system as well as perform static analysis. Using a service like Jenkins helped to regulate the test environment and produce an easy to comprehend output. The decision to use Jenkins as the CI tool of choice was made due to past familiarity with the service, however, on reflection, this could have been done locally just as easily. Jenkins has many features that were not utilised in this project such as project builds. Project builds is a tool that creates a build copy of the software after a set of tests have been run in the pipeline - which can be a useful output for manual testing. As this project was written in Python, there was no build output (as Python does not require compiling in a similar way to other languages such as C). Project builds is just one example of how Jenkins could be considered a heavy weight solution to having a controlled testing environment. Whilst it was a success in this project, it could be replaced by a light weight alternative should the project continue.

\newpage

\section{Improvements}
Keeping a diary throughout the project proved an invaluable resource when writing the report. The dairy helped with managing when development took place, the problems that were discovered and references that needed to be added. Were the project completed again, improving the diary even further would be an asset. As well as writing short bullet points that summarised issues that were dealt with that day, it would have been better to write a small paragraph. These paragraphs could then be used in the report to replace the overview at the start of the iteration. Furthermore, it would be better to reference features by name in the diary so that it can be more easily read without having to reference another document.

To obtain feedback on and improve the UI design, paper copies of a selection of designs could have been taken to the Aberystwyth University Science Week 2017. This would have given the opportunity to see if the target audience liked the proposed design of the UI and what, if anything, they might change. Collating these results then could have led to a better design that would be more fitting for the target audience.


\section{Future work}
There are many sections of the project that still require refactoring. Where this has been done already, the code is of very good standard, but sections of the Charades Game are not written as well as they could be. The main place where this is present is in the views.py file. This file holds all the controller functions for every page (including the API) and could quite easily be simplified through refactoring. The stages that would be done for this include, removing duplicate code used in redirects, moving simple code to other functions for both readability and maintainability, and using multiple files instead of just one to hold all the controller functionality.

An idea for a possible piece of future work is to enable the Charades Game and the Hologram creation system to interact with one another. The design for this would turn the camera (pointed at the actor) on when they were acting, and then display messages when either the actor is choosing a new word or phrase, or a viewer has guessed correctly. This could easily be done, assuming the hologram creation system and the Charades Game server are hosted on the same machine. The hologram creation system could poll a file on the machines hard drive, which contains what the hologram should currently be displaying (e.g. "Camera", "Winner", "Waiting"). The website would then be able to update the file when the game state changes.

Currently the website for the Charades Game is not hosted anywhere online. Deploying this system to either a cloud service or on a server would not be a difficult task. The website can be installed and run using the installation instructions bundled with the project in the README.txt file. This would need to be done to use the charades system at future outreach events.

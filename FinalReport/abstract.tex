\thispagestyle{empty}

\begin{center}
    {\LARGE\bf Abstract}
\end{center}

This report describes the process followed to develop a system to create real-time holograms using the Pepper's Ghost pyramid technique and an accompanying charades game. The system was first proposed to be used at Aberystwyth university's Science week in 2018, but is also suitable to be used at any appropriate outreach events. The Pepper’s ghost pyramid technique is an excellent tool to use in outreach as it is both simple to understand how it works, and creates an impactful display. In order to improve the outreach experience for participants, the charades game is designed to provide a way to interact with the holographic system.

This report will present the purpose, technologies and processes surrounding this system. The development process followed an adapted single person FDD plan driven methodology which will be discussed in detail below. The report will also mirror the methodology and present development tasks and their completion relative to software iterations which took place on a weekly basis. Furthermore, there will be a focus on the software quality and testing performed throughout the development of the process.

Finally, the report will conclude with a critical evaluation of the system itself, the implementation choices and the development processes and methodologies. This will include but not be limited to the programming languages and frameworks used, continuous integration, source control and testing tools, and prototyping and implantation testing.
